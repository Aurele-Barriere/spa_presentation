\documentclass[page number]{beamer}
\usetheme[sectionpage=none,numbering=fraction,progressbar=foot]{metropolis}

\usepackage{pgf,tikz}
\usetikzlibrary{arrows}
\usetikzlibrary{positioning,shapes,fit}
\usepackage{xcolor}
\usepackage{amssymb}
\usepackage{amsmath}

\makeatletter
\makeatother

\setcounter{tocdepth}{1} % remove subsection from table of contents

% colors
\definecolor{mDarkRed}{HTML}{6F1616}
\definecolor{mDarkGreen}{HTML}{106235}
\definecolor{mTeal}{HTML}{112233}
\definecolor{mBlack}{HTML}{000000}
\setbeamercolor{normal text}{fg=mTeal}
\setbeamercolor{alerted text}{fg=mDarkRed}
\setbeamercolor{example text}{fg=mDarkGreen}
\setbeamercolor{title separator}{fg=purple,bg=mBlack}

\def\outline{
  \begin{frame}[plain,noframenumbering]
    \frametitle{Outline}
    \tableofcontents[currentsection]
  \end{frame}
}


\begin{document}
\title[The Power of Clause Learning]{Understanding the Power of Clause Learning}

\author{Paul Beame \hfill Henry Kautz \hfill Ashish Sabharwal}
\vfill
\date{
  \vfill
  Aur\`ele Barri\`ere
  \vfill
  \textbf{03/10/18}}

\def\outline{
  \begin{frame}[plain,noframenumbering]
    \frametitle{Outline}
    \tableofcontents[currentsection]
  \end{frame}
}

\begin{frame}[plain,noframenumbering]
  \vspace{-2cm}
  \maketitle
  \vspace{-4cm}
\end{frame}

%% \metroset{sectionpage=none}

\metroset{sectionpage=progressbar}

\begin{frame}{Context}
  \begin{block}{Published in IJCAI in 2003}
  \end{block}
  \vfill
  \begin{block}{Motivation}
    \begin{itemize}
    \item Deciding SAT as quickly as possible has many uses.
    \item DPLL-based algorithms work well.
    \item The most efficient algorithms are just variants of DPLL with clause learning.
    \item No formal work to explain the success of clause learning.
    \end{itemize}
  \end{block}
\end{frame}

\begin{frame}{Reminders and definitions}
  % SAT and DPLL
  % clause learning
  % General and regular Resolution
  % restarts
\end{frame}

\begin{frame}{Contribution}
  % the 3 points
  % math framework to analyze clause learning
  % characterization of its power
  % ways to improve solver performance?
\end{frame}

\begin{frame}{Comparing SAT methods}
  % shortest proof of UNSAT
  % source
\end{frame}

\begin{frame}{Known Results}
  % see diagram
\end{frame}

\begin{frame}{New Results}
  % see diagram
\end{frame}

\begin{frame}{Learning Schemes}
  % Existing, and FirstNEwCut
\end{frame}

\begin{frame}{Clause Learning can be better than Regular}
  % formulas, implications etc
  % section 5
\end{frame}

\begin{frame}{Clause Learning and General Resolution}
  % section 4
  % equivalent with restarts
\end{frame}

\begin{frame}{Experimental Results}
  % section 6
\end{frame}

\begin{frame}{Conclusion}
\end{frame}

\begin{frame}{Overview}
\end{frame}

\end{document}
